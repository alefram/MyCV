\documentclass{letter} % Use the custom resume.cls style

\usepackage[left=0.4 in,top=0.4in,right=0.4 in,bottom=0.4in]{geometry} % Document margins
\newcommand{\tab}[1]{\hspace{.2667\textwidth}\rlap{#1}}
\newcommand{\itab}[1]{\hspace{0em}\rlap{#1}}
% \name{Alexis Fraudita} % Your name
% % You can merge both of these into a single line, if you do not have a website.
% \address{+58(412) 111-5669  \\ Caracas, Venezuela}
% \address{\href{mailto:fraumalex@gmail.com}{fraumalex@gmail.com} \\ 
% \href{https://www.linkedin.com/in/alefram/}{linkedin} \\ 
% \href{https://github.com/alefram}{GitHub}}  %

\begin{document}

\begin{rSection}{Carta de Presentación} 

    \vspace{0.7cm}

    \begin{tabular}{ @{} >{\bfseries}l @{\hspace{3ex}} l  }
        

    \end{tabular}

% \vspace{1cm}

Estimado Comité de Selección.\\

Mi nombre es Alexis Fraudita y me dirijo a ustedes para expresar mi interés en la beca de doctorado que me fue informada por el investigador Juan Carlos Alvarez Hostos del Instituto CIMNE, asociado al CONICET en Argentina. Estoy emocionado por la posibilidad de contribuir a la investigación en el desarrollo de sistemas electrónicos y control en robótica para la agricultura 5.0, y considero que esta beca es una oportunidad invaluable para avanzar en mis estudios y desarrollo profesional.\\

Soy graduado de la Universidad Central de Venezuela, donde desarrollé un fuerte interés por la robótica. Durante mis estudios, participé como profesor asistente en el laboratorio de Sistemas Digitales y Microcontroladores I, lo que me permitió adquirir habilidades esenciales en programación de microcontroladores y diseño de circuitos digitales. Por otro lado, desarrollé mi tesis de pregrado en inteligencia artificial aplicada a la robótica, donde pude adquirir habilidades en simulación robótica y programación a alto nivel. Estas experiencias han cimentado mi deseo de continuar mi formación académica a nivel doctoral, enfocándome en diseño electrónico e inteligencia artificial aplicada a la robótica.\\ 

La razón por la que solicito esta beca es que los recursos financieros son cruciales para llevar a cabo mis investigaciones, especialmente considerando los costos asociados con equipos y computación. Mi compromiso con la investigación innovadora se alinea perfectamente con los objetivos del programa ANPCyT. Estoy convencido de que esta beca no solo facilitará mi desarrollo académico, sino que también me permitirá contribuir significativamente al campo de la robótica.\\

Agradezco sinceramente su consideración hacia mi solicitud. Estoy entusiasmado por la posibilidad de formar parte de este programa y contribuir a la comunidad académica del CONICET - San Juan y INAUT. Quedo a disposición para proporcionar cualquier información adicional que requieran.\\

\vspace{1cm}

Atentamente, \\
Alexis Fraudita \\
Caracas, Venezuela \\
fraumalex@gmail.com \\
+58(412)1115669

\end{rSection}
\end{document}
